\documentclass[
    twoside,
    twocolumn,
    letterpaper,
    %defaultfont,
    %rmheading,
    10pt]{article}
\usepackage{myfont}
\usepackage{myformat}
\usepackage[backend=biber,
            style=numeric-comp]{biblatex}
\renewcommand*\bibfont{\small}
\addbibresource{References.bib}
\usepackage{listings}
\usepackage{xcolor}
\definecolor{codeforeground}{rgb}{0.22,0.22,0.25}
\definecolor{codebackground}{rgb}{1,1,1}%{0.94,0.94,0.98}
\definecolor{codeblue}{rgb}{0,0.42,0.72}
\definecolor{codepurple}{rgb}{0.28,0.0,0.82}
\lstdefinelanguage{LSQT}{
    classoffset=1,
    morekeywords={Medium, Layer},
    keywordstyle=\color{codeblue},
    classoffset=2,
    morekeywords={
        HenyeyGreenstein, 
        Rayleigh,
        Sggx,
        Null,
        Lambertian,
        MicrosurfaceLambertian,
        MicrosurfaceDielectric,
        MicrosurfaceConductive,
        OrenNayarDiffuse
    },
    keywordstyle=\color{codepurple},
    classoffset=0
}
\newcommand\namett[2]{{\color{code#1}\texttt{#2}}}
\lstset{
    language=LSQT,
    basicstyle=\color{codeforeground}\ttfamily,
    backgroundcolor=\color{codebackground},
    breakatwhitespace=true,
    breaklines=true,
    showspaces=false,
    showstringspaces=false,
    showtabs=false,
    tabsize=2,
    frame=single,
    numberstyle=\footnotesize\ttfamily
}
\raggedbottom
\begin{document}
\title{Layered-SQT usage}
\author{
M. Grady Saunders\\
\texttt{mgs8033@rit.edu}}
\date{}
\maketitle

\section{Introduction}

This document describes \texttt{layered-sqt}, a command-line 
tool for simulating the Bidirectional Scattering Distribution 
Function (BSDF) which emerges from a layered assembly. In this
context, a \emph{layered assembly} is a theoretical construction
consisting of $N$ layers separated by $N+1$ participating media for
$N \ge 1$. A layer is an infinite plane which is offset along (and 
normal to) the $z$-axis, and which is associated with a constituent 
BSDF that describes how light is scattered upon intersection. 
A medium is thought to occupy the space between adjacent layers or, 
in boundary cases, the spaces above and below the top and bottom 
layers respectively. See figure \ref{fig:assembly} for clarification.

The \emph{emergent BSDF} is the BSDF observed in the limit as one backs
infinitely far away from the assembly or, identically, as the assembly
shrinks to an infinitesmial point.
To ensure that the emergent BSDF is well-defined, 
\texttt{layered-sqt} requires that participating media be homogeneous,
i.e., that scattering properties be independent of spatial location. 
Furthermore, for simplicity/tractability, \texttt{layered-sqt} does 
not account for wavelength-dependence.

\subsection{Basics of LSQT format}

The structure of a layered assembly is easy enough to convey 
in plain-text with rudimentary syntax, so this is the format 
\texttt{layered-sqt}
accepts as input. By convention, we refer to this format as
``LSQT format'', and we suffix associated filenames with the extension 
\texttt{.lsqt} (though this suffix is not strictly required for the program
to run). 

An LSQT file is therefore a line-by-line plain-text description of a 
layered assembly from top to bottom. So, the first line describes the 
top medium, the second line describes the top layer beneath the top medium, 
the third line describes the medium beneath the top layer, and so on until 
the bottom medium. 
That being the case, odd-numbered lines describe media
and even-numbered lines describe layers. For media as well as layers, 
the general syntax is 
\begin{verbatim}
Name key1=val1 key2=val2
\end{verbatim}
where \texttt{Name} is described by keyword arguments 
\texttt{key1} and \texttt{key2}---importantly, this syntax is
whitespace-delimited, so keyword arguments of the form \texttt{key=val} 
must not contain whitespace. It is also worth mentioning that 
keyword arguments may appear in any order.

\begin{figure}
\begin{center}
    \includegraphics[width=\columnwidth]{Fig-layers/Fig-layers.pdf}
    \caption{An example diagram of a layered assembly with 3 layers
    and 4 participating media.
    \label{fig:assembly}}
\end{center}
\end{figure}

\subsection{Basics of program usage}

As \texttt{layered-sqt} is a command-line tool, it runs
on the command line, whereby it scans command-line arguments for
(optional) configuration flags and a (required) input filename
appearing somewhere as a positional argument, i.e., an argument 
not consumed by a flag. It then parses the input file, simulates 
the emergent BSDF,
and writes the results to a RAW-format file ready for conversion to
SQT-format via \texttt{raw2sqt}. For instance, 
\begin{verbatim}
$ ./layered-sqt example.lsqt -p 50000 \
  -o example.raw
\end{verbatim}
simulates a layered assembly as described in \texttt{example.lsqt} 
with 50,000 paths, and writes the emergent BSDF in plain-text RAW-format to
\texttt{example.raw}. To convert the BSDF to SQT-format, which is necessary
for use in DIRSIG, run
\begin{verbatim}
$ ./raw2sqt example.raw
\end{verbatim}
which will write a new file \texttt{example.sqt}.

Two of the most common flags appear above, namely
\texttt{-p} (or \texttt{--path-count}) to specify the number of paths used in
the simulation and \texttt{-o} (or \texttt{--output}) to specify the output 
filename. To see a list of all acceptable flags with brief descriptions and 
default values, pass \texttt{-h} (or \texttt{--help}), or simply run 
\texttt{layered-sqt} with no filename. As an aside,
\texttt{layered-sqt} verifies parameters specified in command-line flags
as well as keyword arguments specified in the input LSQT file.
In the event that something has an unreasonable value, the program
issues an error and fails in a controlled manner.

Lastly, \texttt{layered-sqt} recognizes the single dash filename 
``\texttt{-}'' as standard input. So, it is possible to pipe the 
(presumably LSQT format) output of a script into \texttt{layered-sqt} 
directly, if this happens to be convenient. As a trivial example,
\begin{verbatim}
$ cat example.lsqt | \
  ./layered-sqt -p 50000 -o example.raw -
\end{verbatim}
is equivalent to just passing \texttt{example.lsqt}.

\section{Tutorial}

This section provides a series of progressively more interesting
``learn-by-doing'' tutorials, which should naturally introduce the
concepts and features in \texttt{layered-sqt}. To get started, create
a working directory \texttt{lsqt-tutorial} and link the relevant programs
for easy access. From one of the Carlson CIS servers, 
type the following commands:
\begin{verbatim}
$ mkdir lsqt-tutorial && cd lsqt-tutorial
$ TMP_PATH=/cis/phd/mgs8033/layered-sqt/bin
$ ln -s $TMP_PATH/layered-sqt
$ ln -s $TMP_PATH/layered-sqt-lssview
$ ln -s /dirs/pkg/dirsig/bin/raw2sqt
\end{verbatim}

\subsection{Hello, world}
\label{sec:tutorial1}

Let's start with something mind-numbingly simple---a 60\% 
reflective Lambertian surface. First, create and enter a 
sub-directory \texttt{tutorial1}.
\begin{verbatim}
$ mkdir tutorial1 && cd tutorial1
\end{verbatim}
Using whatever text-editing program you prefer, create a 
plain-text LSQT file $\texttt{Lambertian.lsqt}$ with three lines.
\begin{lstlisting}[numbers=left]
Medium
Layer z=0 Lambertian fR=0.6
Medium
\end{lstlisting}
As an aside, LSQT format input is syntax-highlighted in this 
document. However, syntax-highlighting files/add-ons are not
available (yet) for any text editor.

Now, run \texttt{layered-sqt} on \texttt{Lambertian.lsqt}, 
\begin{verbatim}
$ ../layered-sqt Lambertian.lsqt
\end{verbatim}
which should write two new files: \texttt{Lambertian.lsqt.lss} and
\texttt{Lambertian.lsqt.raw}. As stated in the introduction, we may
convert the plain-text RAW file to a binary SQT file for use with 
DIRSIG by running \texttt{raw2sqt},
\begin{verbatim}
$ ../raw2sqt Lambertian.lsqt.raw
\end{verbatim}
which generates the final SQT file \texttt{Lambertian.lsqt.sqt}
suitable for use in a DIRSIG scene.

Now, what is the LSS file?
\emph{LSS} is an initialism for \emph{LSQT-Slice}, which is the
name of the internal file format \texttt{layered-sqt} uses to store 
simulation data, and it is useful 1) for previewing the layered BSDF
\emph{without setting up and rendering a DIRSIG scene}, and 2) for
simulating a layered BSDF progressively, with multiple runs of 
\texttt{layered-sqt}.

\begin{figure}
\begin{center}
    \includegraphics[width=0.75\columnwidth]{tutorial1.png}
    \caption{Image preview of a 60\% reflective Lambertian BRDF 
    output by \texttt{layered-sqt-lssview}, as in tutorial
    \S\ref{sec:tutorial1}.
    \label{fig:tutorial1}}
\end{center}
\end{figure}

Run \texttt{layered-sqt-lssview} to preview the Lambertian BRDF.
\begin{verbatim}
$ ../layered-sqt-lssview Lambertian.lsqt.lss
\end{verbatim}
This should write a new file, \texttt{Lambertian.lsqt.lss.png}, which 
is a 512x512 rendering of the BSDF applied to a ball, shown
in figure \ref{fig:tutorial1}. When connected 
to the CIS server with X-window support (from a Linux machine, log-in 
with \texttt{ssh -X}), this image may be viewed by running \texttt{eog}.
\begin{verbatim}
$ eog Lambertian.lsqt.lss.png
\end{verbatim}

Importantly, \texttt{layered-sqt-lssview} is \emph{not} a full-blown
path-tracer, and may not perfectly represent how the BSDF will appear
in DIRSIG. It only accounts for the direct (first bounce) contributions 
of a few directional light sources, and it further uses tone-mapping and
sRGB correction, such that the output image is not suitable for any
radiometric analysis. The intended use of this preview image is to 
determine if a simulated BSDF is suitably convergent/noise-free.

For more information on the Lambertian BSDF, refer to 
the documentation in \S\ref{sec:doc-layers-lambertian}.

\subsection{Hello, world, with roughness}
\label{sec:tutorial2}

Let's expand on \S\ref{sec:tutorial1} by simulating a 60\% reflective
Lambertian BRDF with a rough microsurface---we assume that, microscopically,
the surface geometry is not perfectly smooth, but is instead characterized by 
a distribution of normals about the primary surface normal. See the 
documentation in \S\ref{sec:doc-layers-microsurface-lambertian} for a 
more detailed explanation.

First, return to the \texttt{lsqt-tutorial} directory, and create
and enter a new subdirectory \texttt{tutorial2}.
\begin{verbatim}
$ cd ..
$ mkdir tutorial2 && cd tutorial2
\end{verbatim}
As before, create a new plain-text LSQT file \texttt{Rough.lsqt} 
with three lines.
\begin{lstlisting}[numbers=left]
Medium
Layer z=0 MicrosurfaceLambertian fR=0.6 alpha=2.4
Medium
\end{lstlisting}
Above, \texttt{fR=0.6} specifies that the Lambertian BRDF is 60\% 
reflective, and \texttt{alpha=2.4} sets the roughness parameter
$\alpha$ to be 2.4---this is very rough. Note that setting 
\texttt{alpha=0} is effectively equivalent to using 
\namett{purple}{Lambertian} with \texttt{fR=0.6} 
and \texttt{fT=0}.

\begin{figure}
\begin{center}
    \includegraphics[width=0.75\columnwidth]{tutorial2.png}
    \caption{Image preview of a 60\% reflective Lambertian BRDF
    with a rough microsurface output by \texttt{layered-sqt-lssview}, as 
    in tutorial \S\ref{sec:tutorial2}.
    \label{fig:tutorial2}}
\end{center}
\end{figure}

Now, run \texttt{layered-sqt} on \texttt{Rough.lsqt} to generate
files \texttt{Rough.lsqt.lss} and \texttt{Rough.lsqt.raw}. This should take
longer to simulate than before, due to the computational complexity of the
microsurface model.
\begin{verbatim}
$ ../layered-sqt Rough.lsqt
\end{verbatim}
When this is complete, use \texttt{layered-sqt-lssview} to generate 
\texttt{Rough.lsqt.lss.png}, shown in figure \ref{fig:tutorial2}.
\begin{verbatim}
$ ../layered-sqt-lssview Rough.lsqt.lss
\end{verbatim}
Notice that the microsurface roughness causes increased reflectance at 
grazing angles, and decreased reflectance elsewhere.

\subsection{A simple substrate}
\label{sec:tutorial3}

Let's now consider a simple two-layer assembly to model a substrate. 
We imagine a 60\% reflective diffuse Lambertian surface with a glossy 
dielectric coating. Before continuing, again return to the 
\texttt{lsqt-tutorial} directory, and create and enter a new 
subdirectory \texttt{tutorial3}.
\begin{verbatim}
$ cd ..
$ mkdir tutorial3 && cd tutorial3
\end{verbatim}

Then create a new plain-text LSQT file \texttt{Substr.lsqt} with
five lines as below.
\begin{lstlisting}[numbers=left]
Medium
Layer z=1 MicrosurfaceDielectric alpha=0.2
Medium eta=1.4
Layer z=0 Lambertian fR=0.6
Medium
\end{lstlisting}
This assembly has two layers, appearing on lines 2 and 4. 
The first layer is positioned at $z$-height 1 with a 
\namett{purple}{MicrosurfaceDielectric}---this is similar to 
\namett{purple}{MicrosurfaceLambertian}, 
but it applies 
the microsurface ideology to the dielectric Fresnel mirror BSDF instead of 
the Lambertian BRDF. This layer forms the surface of our glossy dielectric 
coating.
The second layer is positioned at $z$-height 0 with a 
\namett{purple}{Lambertian} as in tutorial \S\ref{sec:tutorial1}.

The medium between these layers on line 3 is attributed a
refractive index $\eta$ (\texttt{eta}) of $1.4$. Note that 
\emph{every} medium in an LSQT
file has a refractive index, which defaults to 1 if left unspecfied. 
Note also that \namett{purple}{MicrosurfaceDielectric} depends on the
refractive indices of the media above and below the layer to which it is
assigned, as it generalizes the dielectric Fresnel mirror BSDF.

Run \texttt{layered-sqt} as usual and preview the resulting LSS 
file with \texttt{layered-sqt-lssview}.
\begin{verbatim}
$ ../layered-sqt Substr.lsqt
$ ../layered-sqt-lssview Substr.lsqt.lss
\end{verbatim}
Now, upon viewing \texttt{Substr.lsqt.lss.png}, we might suspect that 
something has gone wrong. As shown in figure \ref{fig:tutorial3-1},
there is quite a bit of low frequency noise and artifacting. What is
going on?

Behind the scenes, \texttt{layered-sqt} estimates values of the emergent 
BSDF at a finite number of sample directions using Monte Carlo integration, 
i.e., integration by averaging random evaluations. So, two central parameters 
controlling simulation fidelity are 1) the number of sample directions and 2) 
the number of random evaluations used to estimate the BSDF at each sample 
direction. By default, \texttt{layered-sqt} uses 80 sample directions and
10,000 random evaluations of potential light paths to estimate the 
emergent BSDF. 

We can change the number of directions $\omega_i$ on the command line 
using the flag \texttt{-wi} (or \texttt{--wi-count}), and we can change the
number of paths using the flag \texttt{-p} (or \texttt{--path-count}).
In general, we should think to increase the number of directions
to increase \emph{resolution}, and we should think to increase the number 
of paths to reduce \emph{noise}.

In the case of our simple substrate BRDF, let's increase the number 
of directions from 80 to 300, and the number of paths from 10,000 to 
100,000. Run \texttt{layered-sqt} with the following flags:
\begin{verbatim}
$ ../layered-sqt -R -wi=300 \
  -p=100000 -rp=0.3 Substr.lsqt
\end{verbatim}
This should take a minute or two to compute. Upon previewing the BRDF with
\texttt{layered-sqt-lssview}, we see something much more sensible, as shown
in figure \ref{fig:tutorial3-2}.

The flag \texttt{-R} (or \texttt{--restart}) tells \texttt{layered-sqt}
to ignore the LSS file and restart the simulation from scratch. 
If the LSS file exists, then \texttt{layered-sqt} will ignore all flags except 
\texttt{-p/--path-count} by default, and add this number of paths to the existing
simulation data (it will also display a message to remind us that is doing
this). So, \texttt{-R} is necessary to regenerate sample directions. 

\begin{figure}
\begin{center}
    \includegraphics[width=0.75\columnwidth]{tutorial3-1.png}
    \caption{Image preview of the simple substrate BRDF in tutorial 
    \S\ref{sec:tutorial3}, as output from running \texttt{layered-sqt} with
    default settings. As is evident, the BRDF would benefit from more
    samples (higher resolution), and more simulated paths (less noise).
    \label{fig:tutorial3-1}}
\end{center}
\end{figure}

\begin{figure}
\begin{center}
    \includegraphics[width=0.75\columnwidth]{tutorial3-2.png}
    \caption{Image preview of the simple substrate BRDF in tutorial
    \S\ref{sec:tutorial3}, as output from running \texttt{layered-sqt} with
    higher settings. In particular, this uses 300 incident directions (instead
    of the default 80) and 100,000 paths (instead of the default 10,000).
    \label{fig:tutorial3-2}}
\end{center}
\end{figure}

We have specified the number of sample directions with \texttt{-wi}, but 
it remains a non-trivial problem to determine what these directions ought to 
be---\texttt{layered-sqt} implements 
a presumably novel algorithm to form a so-called \emph{Redundancy Reduced
Sample Set (RRSS)} of directions by initially sampling \emph{more} directions 
than the user requests, then choosing the subset of those directions that 
minimizes a quantitative measure of ``redundancy'' with respect to a desirable 
density, being the normalization of the non-cosine-weighted BSDF. 

The flag \texttt{-rp} (or \texttt{--rrss-path-frac}) thus specifies
the fraction of paths (as specified by \texttt{-p}) used to initially
estimate the emergent BSDF in forming the RRSS of directions $\omega_i$. 
There is also the flag \texttt{-rx} (or \texttt{--rrss-oversampling}), 
to specify the RRSS oversampling multiplier. (We discuss this flag for 
completeness, but in practice it is never really necessary to specify 
\texttt{-rx}, as the default of 4 is generally sufficient.)
For example, 
\begin{verbatim}
$ layered-sqt \
  -wi=100 -p=20000 -rp=0.25 -rx=5 File.lsqt
\end{verbatim}
samples $100\times5=500$ directions initially, and uses
$20000\times0.25=5000$ paths to estimate the emergent BSDF at each of 
these 500 directions, then forms an RRSS containing 100 of these 500 
directions, and lastly uses $20000-5000=15000$ paths to improve the 
estimates of the emergent BSDF at each of these 100 directions (such that
the total number of paths is $20000$).

In general, increasing \texttt{-rp} and \texttt{-rx} leads to better
sample placement, at the (potentially extreme) cost of computation time.
We discuss practical simulation strategies using these flags, including 
incremental simulation with multiple runs of \texttt{layered-sqt}, over
the remaining tutorial sections. 

\subsection{Adding dust}
\label{sec:tutorial4}

Let's now add a layer of back-scattering dust to the surface
of the substrate BRDF in tutorial \S\ref{sec:tutorial3}. As usual, return
to the \texttt{lsqt-tutorial} directory, and create and enter a new 
subdirectory \texttt{tutorial4}.
\begin{verbatim}
$ cd ..
$ mkdir tutorial4 && cd tutorial4
\end{verbatim}

Next create another plain-text LSQT file \texttt{Dusty.lsqt} with three 
layers and thus seven lines.
\begin{lstlisting}[numbers=left]
Medium
Layer z=2 Null
Medium mus=0.2 HenyeyGreenstein g=-0.4
Layer z=1 MicrosurfaceDielectric alpha=0.2
Medium eta=1.4
Layer z=0 Lambertian fR=0.2
Medium
\end{lstlisting}
We attribute a \namett{purple}{Null} to the topmost layer appearing
on line 2, such that this layer only exists to separate media, and 
no scattering happens at the layer itself. The medium above on line 1 
is vacuum as usual. The medium below on line 3 models our layer of dust.
Lines 4 through 7 are just lines 2 through 5 from the previous
section, except we have reduced $f_R$ from 60\% to 20\% to exaggerate
the dust effect.

Similarly to refractive index $\eta$ (\texttt{eta}), \emph{every} 
medium in an LSQT file has two volume scattering parameters---the 
scattering coefficient $\mu_s$ (\texttt{mus}) and the absorption 
coefficient $\mu_a$ (\texttt{mua}), which default to zero if 
left unspecified. Scattering and absorption events happen according
to Beer's Law in a homogeneous medium. That is, the probability of 
scattering within a particular distance $d$ is given by an exponential 
distribution $1 - \exp{(-\mu_s d)}$, where $\mu_s$ is the distribution 
parameter, and analogously for absorbing within a particular distance 
$d$ and $\mu_a$. We thus identify the units of $\mu_s$ and $\mu_a$ as 
``inverse distance'', and we interpret the reciprocals $1/\mu_s$ and 
$1/\mu_a$ as the mean distances between scattering and absorption events
respectively. Note that in the limit $\mu_s = \mu_a = 0$ (the default values),
the mean distances tend to infinity, such that scattering and absorption events
cease to exist.

To simulate a thin layer of back-scattering dust, 
we attribute a scattering coefficient $\mu_s$ (\texttt{mus}) of $0.2$ and 
a \namett{purple}{HenyeyGreenstein} with shape parameter $g$ of $-0.4$
(more on this in a moment). We choose $\mu_s$ to be $0.2$ such that the mean 
distance between scattering events is $1/\mu_s = 5$ units. This is somewhat 
large in comparison to the thickness of the dust layer, which is 1 unit 
(the difference of the $z$-heights of layers on lines 2 and 4), so the dust
appears suitably ``thin''. When plucking this number out of the sky, this is 
the line of reasoning to use. 

The Henyey-Greenstein phase function is widely used in computer graphics
due to its simplicity and intuitive parameterization (although it was 
originally intended to model scattering of interstellar dust clouds). The
phase function is equipped with a single shape parameter $g \in (-1, +1)$,
which is identically the mean scattering cosine. That is, it tends to
forward scatter as $g \to +1$ and back scatter as $g \to -1$. See the
documentation in \S\ref{sec:doc-media-henyey-greenstein} for more
information.

Now run \texttt{layered-sqt} on \texttt{Dusty.lsqt} with the same flags as in
the previous section.
\begin{verbatim}
$ ../layered-sqt -wi=200 \
  -p=100000 -rp=0.2 Dusty.lsqt
\end{verbatim}
Notice that we do not need \texttt{-R} this time because no LSS file 
exists yet. As before, this should take a minute or two to compute.
The image preview as output by \texttt{layered-sqt-lssview} is shown in
figure \ref{fig:tutorial4}.

\begin{figure}
\begin{center}
    \includegraphics[width=0.75\columnwidth]{tutorial4.png}
    \caption{Image preview of the modified simple substrate BRDF with 
    back-scattering
    dust in tutorial \S\ref{sec:tutorial4}, where the bottom Lambertian BRDF 
    layer is reduced to 20\% reflectance to exaggerate the dust effect.
    \label{fig:tutorial4}}
\end{center}
\end{figure}

\section{Documentation}
\label{sec:doc}

This section documents the properties of participating media and 
scattering layers in more detail, and enumerates the available 
scattering models for media and layers, and how to specify them in 
LSQT format. Note that ``phase function'', ``BRDF'', and ``BSDF'' are
specific types of scattering models. In particular,
\begin{itemize}
    \item a \emph{phase function} is a scattering model which accounts
        for volume-scattering within a participating medium,
    \item a \emph{BRDF}, i.e., a \emph{Bidirectional Reflectance Distribution 
        Function}, is a scattering model which accounts for only reflection at 
        a surface, and
    \item a \emph{BSDF}, i.e., a \emph{Bidirectional Scattering Distribution 
        Function}, is a scattering model which accounts for reflection and 
        transmission at a surface.
\end{itemize}

Note that phase functions are normalized by 
definition, as volume-absorbtion is accounted for by a separate parameter. 
BRDFs and BSDFs, however, account for absorption as well as scattering, 
and so are only normalized if the model is non-absorbing (in other words,
perfectly energy conserving). To state all of this more rigorously, a
phase function $p$ is normalized with respect to integration over 
incident directions $\omega_i$, such that
\begin{equation*}
    \int_{\mathcal{S}^2} p(\omega_o\to\omega_i) \diff{\omega_i} = 1.
\end{equation*}
However, a BRDF/BSDF $f$ is only normalized as such if it is 
non-absorbing,
\begin{equation*}
    \int_{\mathcal{S}^2} f(\omega_o\to\omega_i) \diff{\omega_i} = 1
    \iff \text{$f$ is non-absorbing.}
\end{equation*}

As this ought to suggest---this document follows the convention that BRDFs 
and BSDFs contain an implicit cosine-weighting with respect to incident
direction. This is often written explicitly elsewhere in the literature,
such that the above normalization condition is written as
\begin{equation*}
    \int_{\mathcal{S}^2} f(\omega_o\to\omega_i) \verts{\cos{\theta_i}}
    \diff{\omega_i} = 1.
\end{equation*}
We effectively make the substitution $f\gets f|{\cos{\theta_i}}|$.

\subsection{Participating media}
\label{sec:doc-media}

A medium is characterized by its 
absolute index of refraction $\eta > 0$, its absorption coefficient 
$\mu_a \ge 0 $, its scattering coefficient $\mu_s \ge 0$, and a local
phase function model.

To specify a medium, simply type the name \namett{blue}{Medium} followed
by (optional) keywords arguments \texttt{eta}, \texttt{mua}, and
\texttt{mus} for $\eta$, $\mu_a$, and $\mu_s$ respectively. Keyword 
arguments for a medium have default values \texttt{eta=1}, \texttt{mua=0}, 
\texttt{mus=0}, such that any may be omitted for brevity.
If $\mu_s > 0$, the keyword arguments must be followed by the name 
of the desired phase function, followed by any keyword arguments for the 
phase function. If $\mu_s = 0$, it is not necessary to specify a phase function
since no scattering occurs.

As the defaults may suggest, \texttt{layered-sqt} considers that vacuuum 
is just a medium with refractive index 1 which neither absorbs nor scatters. 
So,
\begin{lstlisting}
Medium
\end{lstlisting}
specifies vacuum. 
% TODO No longer required?
% Note---the implementation requires that top and 
% bottom media be non-absorbing and non-scattering, such that $\mu_a=\mu_s=0$,
% and issues an error if this requirement is not met.

\subsubsection{Henyey-Greenstein phase}
\label{sec:doc-media-henyey-greenstein}

The Henyey-Greenstein phase function is given by 
\begin{equation*}
    p(\omega_o\to\omega_i) = 
    \frac{1}{4\pi}
    \frac{1-g^2}{(1+g^2+2g\omega_o\cdot\omega_i)^{3/2}}
\end{equation*}
with shape parameter $g\in(-1,1)$. It may be helpful to note 
that $g$ is the mean cosine of the scattered direction with respect to the 
incident direction, such that $p$ becomes forward scattering as $g\to1$, 
back scattering as $g\to-1$, and uniformly/isotropically scattering 
as $g\to0$. See d'Eon's write-up
\cite[p.~19]{dEon:16} for more information.

To specify the Henyey-Greenstein phase in LSQT format, use the name
\namett{purple}{HenyeyGreenstein} followed by the (optional) keyword 
argument \texttt{g} for $g$, which is zero by default. For example,
\begin{lstlisting}
Medium mus=1 HenyeyGreenstein g=-0.22
\end{lstlisting}
specifies a moderately back-scattering medium.

\subsubsection{Rayleigh phase}
\label{sec:doc-media-rayleigh}

The general form of the Rayleigh phase function is given by
\begin{equation*}
    p(\omega_o\to\omega_i) = 
    \frac{3}{16\pi}
    \bracks{
        \frac{1 + 3\gamma}{1 + 2\gamma} +
        \frac{1 -  \gamma}{1 + 2\gamma} (\omega_o\cdot\omega_i)^2}
\end{equation*}
where
\begin{equation*}
    \gamma = \frac{\rho}{2 - \rho}
\end{equation*}
where, in turn, $\rho \in [-1, 1]$ is a shaping parameter known as 
the \emph{depolarization factor}. The simpler, more common form of the
Rayleigh phase function is given by
\begin{equation*}
    p(\omega_o\to\omega_i) = \frac{3}{16\pi}(1 + (\omega_o\cdot\omega_i)^2),
\end{equation*}
which is obtained by setting $\rho = 0$. See d'Eon's write-up
\cite[p.~19]{dEon:16} for more information.

The Rayleigh phase function prefers parallel scattering
to perpendicular scattering. That is to say, scattering in directions 
perpendicular to the direction of propagation is diminished, and scattering 
is symmetric in terms of forward- versus back-scattering. This effect is 
maximized as $\rho \to -1$ and minimized as $\rho \to +1$. Note that $p$ 
becomes uniformly/isotropically scattering in the limiting case $\rho = +1$.

To specify the Rayleigh phase function in LSQT format, use the name 
\namett{purple}{Rayleigh} followed by the (optional) keyword argument
\texttt{rho} for $\rho$, which is zero by default. For example,
\begin{lstlisting}
Medium mus=1 Rayleigh rho=-0.5
\end{lstlisting}
specifies a medium with an exaggerated ``Rayleigh effect'' relative to the
common form where $\rho=0$.

\subsubsection{SGGX phase}
\label{sec:doc-media-sggx}

The Symmetric GGX (SGGX) phase function, given by 
Heitz et al. \cite{Heitz:15}, is based on \emph{microflake
theory}, wherein we assume volume scattering happens due to surface 
scattering with infinitely many microscopic disjoint flakes. 
The general form of the phase function is 
\begin{equation*}
    p(\omega_o \to \omega_i) = 
    \int_{\mathcal{S}^2} p_m(\omega_m, \omega_o \to \omega_i) 
    D_{\omega_o}(\omega_m) \diff{\omega_m}
\end{equation*}
where $p_m$ is the microscopic phase function 
assigned to each flake, and $D_{\omega_o}$ is the distribution of
\emph{visible} flake normals for a particular viewing/outgoing direction 
$\omega_o$.

The distribution of flake normals $D$ is defined by its projected areas with 
respect to three orthogonal directions. In particular,
\begin{equation*}
    D(\omega_m) = 
    \frac{1}{\pi\sqrt{\det\mathbf{S}}(\omega_m\cdot\mathbf{S}^{-1}\omega_m)^2}
\end{equation*}
where $\mathbf{S}$ is a $3\times3$ real symmetric matrix. The eigenvalues of
$\mathbf{S}$ are squared projected areas, and the eigenvectors of $\mathbf{S}$
and the orthogonal coordinate directions. The distribution of \emph{visible}
normals $D_{\omega_o}$ is obtained from $D$ as
\begin{align*}
    D_{\omega_o}(\omega_m) &= 
        \frac{D(\omega_m)}{\sigma(\omega_o)}
        \Angles{\omega_o, \omega_m}
    \intertext{where $\sigma$ yields projected area along $\omega_o$ and}
    \Angles{\omega_o, \omega_m} &= 
    \begin{cases}
        \omega_o \cdot \omega_m & 
        \text{$(\omega_o, \omega_m)$ in same hemisphere,}\\
        0 & \text{otherwise.}
    \end{cases}
\end{align*}
So, unlike the phase functions in the previous sections, the SGGX phase
function has a global orientation, meaning that $p$ depends on the specific
values of directions $\omega_o$ and $\omega_i$, and not just
on the angle between them. 

Due to the constraint in \texttt{layered-sqt} that the emergent BSDF must be
isotropic, we cannot allow all configurations of $\mathbf{S}$. If $\mathbf{S}$
is written as the eigendecomposition
$\mathbf{S} = \mathbf{Q}^\top \mathbf{\Lambda} \mathbf{Q}$,
then 1) we fix $\mathbf{Q} = \mathbf{I}$, the identity matrix, and we 2) require
that the eigenvalues of $\mathbf{\Lambda}$ describing projected area in the
XY plane be identical. We thus obtain
\begin{equation*}
    \mathbf{S} = \begin{bmatrix}
        A_{\parallel}^2 & 0 & 0 \\
        0 & A_{\parallel}^2 & 0 \\
        0 & 0 & A_{\perp}^2 
    \end{bmatrix}
\end{equation*}
with free parameters $A_{\parallel} > 0$ and $A_{\perp} > 0$.

To specify the SGGX phase function in LSQT format, use the name
\namett{purple}{Sggx} followed by (optional) keyword arguments
\texttt{Apara} for $A_{\parallel}$, \texttt{Aperp} for $A_{\perp}$, and
\texttt{type} for the micro-phase function type (either \texttt{Specular} 
or \texttt{Diffuse}). For example,
\begin{lstlisting}
Medium mus=1 Sggx Apara=0.5 type=Diffuse
\end{lstlisting}
By default, \texttt{Apara=1}, \texttt{Aperp=1}, and 
\texttt{type=Specular}. Note that only the ratio of
$A_{\parallel}$ to $A_{\perp}$ matters, so it always acceptable to fix
$A_{\perp} = 1$ and just specify $A_{\parallel}$ (or vice versa).

\subsection{Layers}
\label{sec:doc-layers}

A layer is characterized by its $z$-height and 
a local BSDF model. As different BSDF models require different 
parameters, there is a different ``type'' of layer for each 
BSDF model implemented in \texttt{layered-sqt}. 

Every layer accepts $z$-height as a parameter, so every line describing a 
layer must begin with the name \namett{blue}{Layer} followed by the (required)
keyword argument \texttt{z}. This is followed in turn by the name of the 
desired BSDF, followed by any keyword arguments for the BSDF. For example,
\begin{lstlisting}
Layer z=1 Null
\end{lstlisting}
specifies a layer at $z$-height 1 with a null BSDF. As explained in 
the next sub-section, a null BSDF is a special case which accepts no additional
keyword arguments.

The implementation requires that layers appear in top-to-bottom order,
such that the $z$-heights of subsequent layers in an LSQT file are
strictly decreasing. If this condition 
is violated, the implementation issues an error and exits.

\subsubsection{Null BSDF}
\label{sec:doc-layers-null}

At times, it is desirable to separate participating media 
\emph{without} scattering at a layer. This is possible in
\texttt{layered-sqt} by assigning a null BSDF with name 
\namett{purple}{Null}, which may be interpreted as a 100\% 
transmissive directional delta function in the direction a 
ray is already traveling. This is useful, for example, to model a 
layer of dust on top of a surface. To do so, we might specify the 
following LSQT file
\begin{lstlisting}[numbers=left]
Medium
Layer z=1 Null
Medium mus=1 HenyeyGreenstein g=0.2
Layer z=0 Lambertian fR=1
Medium
\end{lstlisting}
which describes a layer of forward-scattering ``dust'' on top 
of a 100\% reflective Lambertian surface.

\begin{figure*}
\begin{center}
    \includegraphics[width=0.75\columnwidth]{example1.png}%
    \hspace{0.05\linewidth}%
    \includegraphics[width=0.75\columnwidth]{example2.png}
    \caption{A sphere with an LSQT BRDF on a Lambertian ground plane 
        rendered from above with DIRSIG5. The LSQT BRDF models a
        dusty dielectric substrate---from top to bottom, we have a 
        vacuum medium, a null BSDF layer, a slightly back-scattering 
        medium with $g=-0.2$, a decently smooth microsurface dielectric 
        BSDF layer, a transparent medium with $\eta=1.5$, 
        and a Lambertian BRDF layer. On the left, the Lambertian
        layer has $f_R=0.8$. On the right, the Lambertian layer has 
        $f_R=0.4$, such that the backscattering due to ``dust'' is
        more obvious. All other parameters are the same in both images.}
\end{center}
\end{figure*}

\subsubsection{Lambertian BSDF}
\label{sec:doc-layers-lambertian}

The Lambertian BSDF models a uniformly scattering surface which 
may reflect, transmit, or both. That is, the (cosine-weighted)
scattering function is
\begin{equation*}
    f(\omega_o\to\omega_i) =
    \frac{|\cos{\theta_i}|}{\pi}
    \begin{cases}
        f_R & \text{$(\omega_o, \omega_i)$ in same hemisphere}\\
        f_T & \text{otherwise}
    \end{cases}
\end{equation*}
where $f_R$ and $f_T$ specify the amount of incident energy 
reflected and transmitted respectively. Both $f_R$ and $f_T$ should
be non-negative numbers such that $f_R + f_T \le 1$ for this to be
physically plausible. In the event that $f_R + f_T = 1$, this is
perfectly energy-conserving.

To specify the Lambertian BSDF in LSQT format, use the name
\namett{purple}{Lambertian} followed by (optional) keyword arguments 
\texttt{fR} and \texttt{fT} for $f_R$ and $f_T$ respectively. 
For example,
\begin{lstlisting}
Layer z=2.2 Lambertian fR=0.8 fT=0.1
\end{lstlisting}
specifies a Lambertian surface at $z$-height 2.2 which is 80\% reflective,
10\% transmissive, and 10\% absorptive. By default, \texttt{fR=1} and 
\texttt{fT=0}. %The implementation requires that the values of \texttt{fR} and 
%\texttt{fT} be physically plausible, i.e., non-negative numbers whose sum does 
%not exceed 1. 

\subsubsection{Microsurface Lambertian BSDF}
\label{sec:doc-layers-microsurface-lambertian}

A microsurface is thought 
to be an infinitesimally thin cloud of microfacets, where each facet 
is thought to scatter light according to another, simpler BSDF. The 
cloud is characterized geometrically by 1) a distribution of the
slopes of the facets and 2) a distribution of the heights of the facets,
which are assumed to uncorrelated, such that the facets are discontinuous. 
The distribution of slopes is parameterized by its so-called roughness
$\alpha$ (more generally, anisotropic roughness $\alpha_x$, $\alpha_y$).
As $\alpha\to\infty$, the distribution of slopes widens and the emergent
BSDF appears rougher. As $\alpha\to0$, the distribution of slopes collapses,
recovering the initial, simpler BSDF in the limiting case $\alpha=0$.

As given in \cite{Heitz:16},
the microsurface Lambertian BSDF is a multiple scattering (stochastic)
microfacet model wherein facets are assumed to be Lambertian scatterers.
As one might expect, this is particularly useful for modeling rough diffuse 
surfaces. The implementation in \texttt{layered-sqt} is parameterized by
roughness $\alpha>0$ and Lambertian BSDF coefficients
$f_R \in [0,1]$ and $f_T \in [0,1]$, such that $f_R + f_T \le 1$.
For reference, $\alpha < 0.1$ is very smooth, 
$0.1 < \alpha < 0.8$ is somewhat rough, and $0.8 < \alpha$ is very rough.

To specify the microsurface Lambertian BSDF 
in LSQT format, use the name \namett{purple}{MicrosurfaceLambertian}
followed by keyword arguments \texttt{alpha} for $\alpha$,
\texttt{fR} for $f_R$, and \texttt{fT} for $f_T$. By default, 
\texttt{alpha=0.5}, \texttt{fR=1}, and \texttt{fT=0}. 
Furthermore, multiple scattering interactions may be disabled by the 
keyword argument
\begin{lstlisting}
use_multiple_scattering=false
\end{lstlisting}
in which case the implementation computes single 
scattering interactions with the microsurface only---this may be
desirable to speed up calculations.

It is 
important to note that, if multiple-scattering interactions are 
disabled, then $f_R + f_T = 1$ does \emph{not} guarantee that the BSDF 
is perfectly energy conserving. 
For small roughness values (say, $\alpha < 0.2$), the energy carried by 
multiple scattering iterations is insignificant, so this may not be a
concern. For greater roughness values however, multiple scattering is important
to prevent energy loss.

\subsubsection{Microsurface dielectric BSDF}
\label{sec:doc-layers-microsurface-dielectric}

The microsurface dielectric BSDF follows the same logical construction as 
the microsurface Lambertian BSDF, except the constituent BSDF assigned to the
facets is the delta Fresnel mirror BSDF. The implementation in 
\texttt{layered-sqt} is parameterized by roughness $\alpha > 0$, a scaling
factor for the Fresnel mirror BRDF $k_R \in [0, 1]$, and a scaling factor for 
the Fresnel mirror BTDF $k_T \in [0, 1]$.

To specify the microsurface dielectric BSDF 
in LSQT format, use the name
\namett{purple}{MicrosurfaceDielectric}
followed by keyword arguments \texttt{alpha}, \texttt{kR}, and
\texttt{kT} for $\alpha$, $k_R$, and $k_T$ respectively. By default, 
\texttt{alpha=0.5}, \texttt{kR=1}, and \texttt{kT=1}.

Unlike the Lambertian case, the single-scattering term is directly 
computable due to the delta function 
in the dielectric Fresnel BSDF. This leads to noticeably faster
simulations, so much so that multiple-scattering is disabled by 
default. Energy loss is less noticeable than the Lambertian case
for moderately rough surfaces,
but is still an issue for very rough surfaces 
(say $\alpha > 0.6$). As such, multiple-scattering may be 
enabled at the user's discretion by the keyword argument
\begin{lstlisting}
use_multiple_scattering=true
\end{lstlisting}
to ensure proper energy conservation.


\subsubsection{Microsurface conductive BRDF}
\label{sec:doc-layers-microsurface-conductive}

TODO

\subsubsection{Oren-Nayar diffuse BRDF}
\label{sec:doc-layers-oren-nayar-diffuse}

TODO

{
\nocite{*}
\raggedright
\printbibliography
}
\end{document}
