\documentclass[
    twoside,
    twocolumn,
    letterpaper,
    %defaultfont,
    %rmheading,
    10pt]{article}
\usepackage{myfont}
\usepackage{myformat}
\usepackage[backend=biber,
            style=numeric-comp]{biblatex}
\renewcommand*\bibfont{\small}
\addbibresource{References.bib}
\raggedbottom
\begin{document}
\title{Layered-SQT usage}
\author{
M. Grady Saunders\\
\texttt{mgs8033@rit.edu}}
\date{}
\maketitle

\section{Introduction}

This document describes \texttt{layered-sqt}, a command-line 
tool for simulating the Bidirectional Scattering Distribution 
Function (BSDF) which emerges from a layered assembly. In this
context, a \emph{layered assembly} is a theoretical construction
consisting of $N$ layers separated by $N+1$ participating media for
$N \ge 1$. A layer is an infinite plane which is offset along (and 
normal to) the $z$-axis, and which is associated with a constituent 
BSDF that describes how light is scattered upon intersection. 
A medium is thought to occupy the space between adjacent layers or, 
in boundary cases, the spaces above and below the top and bottom 
layers respectively. 
To ensure that the emergent BSDF is well-defined, 
\texttt{layered-sqt} requires that participating media be homogeneous,
i.e., that scattering properties be independent of spatial location. 
Furthermore, for simplicity/tractability, \texttt{layered-sqt} does 
not account for wavelength-dependence.

\subsection{Basics of LSQT format}

The structure of a layered assembly is easy enough to convey 
in plain-text with rudimentary syntax, so this is the format 
\texttt{layered-sqt}
accepts as input. By convention, we refer to this format as
``LSQT format'', and we suffix associated filenames with the extension 
\texttt{.lsqt} (though this suffix is not strictly required for the program
to run). 

An LSQT file is therefore a line-by-line plain-text description of a 
layered assembly from top to bottom. So, the first line describes the 
top medium, the second line describes the top layer beneath the top medium, 
the third line describes the medium beneath the top layer, and so on until 
the bottom medium. That being the case, odd-numbered lines describe media
and even-numbered lines describe layers. For media as well as layers, 
the general syntax is 
\begin{verbatim}
Name key1=val1 key2=val2
\end{verbatim}
where \texttt{Name} is described by keyword arguments 
\texttt{key1} and \texttt{key2}---importantly, this syntax is
whitespace-delimited, so keyword arguments of the form \texttt{key=val} 
must not contain whitespace. It is also worth mentioning that 
keyword arguments may appear in any order.

\subsection{Basics of program usage}

As \texttt{layered-sqt} is a command-line tool, it runs
on the command line, whereby it scans command-line arguments for
(optional) configuration flags and a (required) input filename
appearing somewhere as a positional argument, i.e., an argument 
not consumed by a flag. It then parses the input file, simulates 
the emergent BSDF,
and writes the results to a RAW-format file ready for conversion to
SQT-format via \texttt{raw2sqt}. For instance, 
\begin{verbatim}
$ ./layered-sqt example.lsqt -p 50000 \
  -o example.raw
\end{verbatim}
simulates a layered assembly as described in \texttt{example.lsqt} 
with 50,000 paths, and writes the emergent BSDF in plain-text RAW-format to
\texttt{example.raw}. To convert the BSDF to SQT-format, which is necessary
for use in DIRSIG, run
\begin{verbatim}
$ ./raw2sqt example.raw
\end{verbatim}
which will write a new file \texttt{example.sqt}.

Two of the most common flags appear above, namely
\texttt{-p} (or \texttt{--path-count}) to specify the number of paths used in
the simulation and \texttt{-o} (or \texttt{--output}) to specify the output 
filename. To see a list of all acceptable flags with brief descriptions and 
default values, pass \texttt{-h} (or \texttt{--help}), or simply run 
\texttt{layered-sqt} with no filename. As an aside,
\texttt{layered-sqt} verifies parameters specified in command-line flags
as well as keyword arguments specified in the input LSQT file.
In the event that something has an unreasonable value, the program
issues an error and fails in a controlled manner.

Lastly, \texttt{layered-sqt} recognizes the single dash filename 
``\texttt{-}'' as standard input. So, it is possible to pipe the 
(presumably LSQT format) output of a script into \texttt{layered-sqt} 
directly, if this happens to be convenient. As a trivial example,
\begin{verbatim}
$ cat example.lsqt | \
  ./layered-sqt -p 50000 -o example.raw -
\end{verbatim}
is equivalent to just passing \texttt{example.lsqt}.

\section{Participating media}

A medium is characterized by its 
absolute index of refraction $\eta \ge 1$, its absorption coefficient 
$\mu_a \ge 0 $, its scattering coefficient $\mu_s \ge 0$, and a 
Henyey-Greenstein phase function
\begin{equation*}
    p(\omega_o\to\omega_i) = 
    \frac{1}{4\pi}
    \frac{1-g^2}{\parens{1+g^2+2g\omega_o\cdot\omega_i}^{3/2}}
\end{equation*}
with shape parameter $g\in(-1,1)$. It may be helpful to note 
that $g$ is the mean cosine of the scattered direction with respect to the 
incident direction, such that $p$ becomes forward scattering as $g\to1$, 
backward scattering as $g\to-1$, and uniformly/isotropically scattering 
as $g\to0$. Of course, the Henyey-Greenstein phase is neither the only 
nor the canonical phase function. However, it is intuitive and often 
good enough to reproduce phenenomenology. So, at least for the time being, 
it is hard-coded into \texttt{layered-sqt}.

\subsection{LSQT format}

To specify a medium, simply type the name \texttt{Medium} followed 
by (optional) keywords arguments \texttt{eta}, \texttt{mua}, 
\texttt{mus}, and \texttt{g} for $\eta$, $\mu_a$, $\mu_s$, and $g$
respectively. Keyword 
arguments for a medium have default values \texttt{eta=1}, \texttt{mua=0}, 
\texttt{mus=0}, \texttt{g=0}, such that any may be omitted for brevity.
So, for example,
\begin{verbatim}
Medium eta=1.2 mus=0.7 g=-0.2
\end{verbatim}
specifies a medium with refractive index 1.2 which is non-absorbing and 
somewhat back-scattering. As the defaults may suggest, \texttt{layered-sqt}
considers that vacuuum is just a medium with refractive index 1 which neither
absorbs nor scatters. So,
\begin{verbatim}
Medium
\end{verbatim}
specifies vacuum. Note---the implementation requires that top and 
bottom media be non-absorbing and non-scattering, such that $\mu_a=\mu_s=0$,
and issues an error if this requirement is not met.

\section{Layers}

A layer is characterized by its $z$-height and 
a local BSDF model. As different BSDF models require different 
parameters, there is a different ``type'' of layer for each 
BSDF model implemented in \texttt{layered-sqt}. 

\subsection{LSQT format (common to all layers)}

Every layer accepts $z$-height as a parameter, so every line describing a 
layer must begin with the name \texttt{Layer} followed by the (required) 
keyword argument \texttt{z}. This is followed in turn by the name of the 
desired BSDF, followed by any keyword arguments for the BSDF. For example,
\begin{verbatim}
Layer z=1 NullBsdf
\end{verbatim}
specifies a layer at $z$-height 1 with a null BSDF. As explained in 
the next sub-section, a null BSDF is a special case which accepts no additional
keyword arguments.

% TODO Must be in top-to-bottom order.

\subsection{Null BSDF}

It is occasionally desirable to separate media 
\emph{without} scattering at a layer. This is possible in
\texttt{layered-sqt} by assigning a null BSDF (with name 
\texttt{NullBsdf}), which may be interpreted as a 100\% transmissive 
directional delta function aligned to the direction a ray is already 
traveling. This is useful, for example, to model a layer of dust on 
top of a surface. To do so, we might specify the following
LSQT file
\begin{verbatim}
Medium
Layer z=1 NullBsdf
Medium g=0.2 mus=1
Layer z=0 LambertianBsdf fR=1 fT=0
Medium
\end{verbatim}
which describes a layer of forward-scattering ``dust'' on top 
of a 100\% reflective Lambertian surface.

\begin{figure*}
\begin{center}
    \includegraphics[width=0.4\linewidth]{example1.png}\hspace{0.05\linewidth}%
    \includegraphics[width=0.4\linewidth]{example2.png}
    \caption{A sphere with an LSQT BRDF on a Lambertian ground plane 
        rendered from above with DIRSIG5. The LSQT BRDF models a
        dusty dielectric substrate---from top to bottom, we have a 
        vacuum medium, a null BSDF layer, a slightly back-scattering 
        medium with $g=-0.2$, a decently smooth microsurface dielectric 
        BSDF layer, a transparent medium with $\eta=1.5$, 
        and a Lambertian BRDF layer. On the left, the Lambertian
        layer has $f_R=0.8$. On the right, the Lambertian layer has 
        $f_R=0.4$, such that the backscattering due to ``dust'' is
        more obvious. All other parameters are the same in both images.}
\end{center}
\end{figure*}

\subsection{Lambertian BSDF}

The Lambertian BSDF models a uniformly scattering surface which 
may reflect, transmit, or both. That is, the (cosine-weighted)
scattering function is
\begin{equation*}
    f(\omega_o\to\omega_i) =
    \frac{|\cos{\theta_i}|}{\pi}
    \begin{cases}
        f_R & \text{$(\omega_o, \omega_i)$ in same hemisphere}\\
        f_T & \text{otherwise}
    \end{cases}
\end{equation*}
where $f_R$ and $f_T$ specify the amount of incident energy 
reflected and transmitted respectively. Both $f_R$ and $f_T$ should
be non-negative numbers such that $f_R + f_T \le 1$ for this to be
physically plausible. In the event that $f_R + f_T = 1$, this is
perfectly energy-conserving.

\subsubsection{LSQT format}

To specify the Lambertian BSDF in LSQT format, use the name
\texttt{LambertianBsdf} followed by keyword arguments \texttt{fR}
and \texttt{fT} for $f_R$ and $f_T$ respectively. The implementation
requires that the values of \texttt{fR} and \texttt{fT} be physically
plausible, i.e., non-negative numbers whose sum does not exceed 1.
For example,
\begin{verbatim}
Layer z=2.2 LambertianBsdf fR=0.8 fT=0.1
\end{verbatim}
specifies a Lambertian surface at $z$-height 2.2 which is 80\% reflective,
10\% transmissive, and 10\% absorptive. 

\subsection{Microsurface Lambertian BRDF}

A microsurface is thought 
to be an infinitesimally thin cloud of microfacets, where each facet 
is thought to scatter light according to another, simpler BSDF. The 
cloud is characterized geometrically by 1) a distribution of the
slopes of the facets and 2) a distribution of the heights of the facets,
which are assumed to uncorrelated, such that the facets are discontinuous. 
The distribution of slopes is parameterized by its so-called roughness
$\alpha$ (more generally, anisotropic roughness $\alpha_x$, $\alpha_y$).
As $\alpha\to\infty$, the distribution of slopes widens and the emergent
BSDF appears rougher. As $\alpha\to0$, the distribution of slopes collapses,
recovering the initial, simpler BSDF in the limiting case $\alpha=0$.

As given in \cite{Heitz:16},
the microsurface Lambertian BRDF is a multiple scattering (stochastic)
microfacet model wherein facets are assumed to be Lambertian reflectors. 
This is particularly useful for representing rough diffuse surfaces. The 
implementation in \texttt{layered-sqt} is parameterized by
roughness $\alpha>0$, the Lambertian BRDF coefficient 
$f_R \in [0,1]$, and the number of local stochastic process 
iterations $n_{\text{iter}} \ge 1$. For rougher microsurfaces 
(say, $\alpha > 0.4$), it is typically more efficient to increase 
$n_{\text{iter}}$ rather than the global path count.

\subsubsection{LSQT format}

To specify the microsurface Lambertian BRDF in LSQT format, use the 
name 
\begin{verbatim}
MicrosurfaceLambertianBsdf
\end{verbatim}
followed by (optional) keyword arguments 
\texttt{alpha}, \texttt{fR}, and \texttt{iter\_count} for $\alpha$,
$f_R$, and $n_{\text{iter}}$ respectively. By default, \texttt{alpha=0.5},
\texttt{fR=1}, and \texttt{iter\_count} is chosen to be 1, 2, 4, or 6 
depending on \texttt{alpha}. Furthermore, multiple scattering may be 
disabled by the keyword argument
\begin{verbatim}
use_multiple_scattering=false
\end{verbatim}
in which case only the single scattering term is computed. It is 
important to note that, if multiple-scattering interactions are 
ignored, then setting \texttt{fR=1} does not guarantee that the BRDF 
is perfectly energy conserving. 
For small roughness values (say, $\alpha < 0.05$), the energy carried by 
multiple scattering interations is insignificant, and so this may not be a
concern. For large roughness values however, multiple scattering is important
to prevent energy loss.

\subsection{Microsurface dielectric BSDF}

The microsurface dielectric BSDF follows the same logical construction as 
the microsurface Lambertian BRDF, except the constituent BSDF assigned to the
facets is the delta Fresnel mirror BSDF. The implementation in 
\texttt{layered-sqt} is parameterized by roughness $\alpha > 0$, a scaling
factor for the Fresnel mirror BRDF $k_R \in [0, 1]$, a scaling factor for 
the Fresnel mirror BTDF $k_T \in [0, 1]$, and the number of local stochastic 
process iterations $n_{\text{iter}} \ge 1$.

\subsubsection{LSQT format}
To specify the microsurface dielectric BSDF in LSQT format, use the name
\begin{verbatim}
MicrosurfaceDielectricBsdf
\end{verbatim}
followed by (optional) keyword arguments \texttt{alpha}, \texttt{kR},
\texttt{kT}, and \texttt{iter\_count} for $\alpha$, $k_R$, $k_T$, and 
$n_{\text{iter}}$ respectively. By default, \texttt{alpha=0.5},
\texttt{kR=1}, \texttt{kT=1}, and \texttt{iter\_count} is chosen to 
be 1, 2, 4, or 6 depending on \texttt{alpha}. As in the Lambertian case,
multiple-scattering may be disabled by the keyword argument
\begin{verbatim}
use_multiple_scattering=false
\end{verbatim}
so that only the single-scattering term is computed. Unlike the
Lambertian case however, the single-scattering term is directly 
computable (without a stochastic process) due to the delta function 
in the Fresnel mirror BSDF---such that disabling multiple-scattering makes
\texttt{iter\_count} irrelevant. This leads to noticeably faster
simulations, though still at the cost of significant
energy loss for rough surfaces.

\subsection{Oren-Nayar diffuse BRDF}

TODO

{
\nocite{*}
\raggedright
\printbibliography
}
\end{document}
